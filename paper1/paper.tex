\documentclass[conference]{IEEEtran} 
\IEEEoverridecommandlockouts
\usepackage{epsfig,graphicx,subcaption}
% \usepackage[hyphens]{url}
\usepackage{hyperref}


%\documentclass[a4paper,10pt]{article}
\usepackage[utf8]{inputenc}
\usepackage{times}

%opening
\title{Placeholder LAVA Title
  \thanks{This work is sponsored by the Assistant Secretary of Defense
    for Research \& Engineering under Air Force Contract
    \#FA8721-05-C-0002.  Opinions, interpretations, conclusions and
    recommendations are those of the author and are not necessarily
    endorsed by the United States Government.} }

\author{
\IEEEauthorblockN{Brendan Dolan-Gavitt\IEEEauthorrefmark{1}, Patrick Hulin\IEEEauthorrefmark{2}, Tim Leek\IEEEauthorrefmark{2}, Ryan Whelan\IEEEauthorrefmark{2}}
\\
\small (Authors listed alphabetically) \\
\\
\IEEEauthorblockA{\IEEEauthorrefmark{1}NYU\\brendandg@nyu.edu}
\IEEEauthorblockA{\IEEEauthorrefmark{2}MIT Lincoln Laboratory\\
\{patrick.hulin,tleek,rwhelan\}@ll.mit.edu}
}

\begin{document}

\maketitle

\begin{abstract}

Work on finding bugs in programs has long been hampered by a shortage of ground-truth corpora on which to evaluate tools.
We present LAVA, a system for automatically injecting large numbers of bugs into existing open-source programs.
Using LAVA, we have injected tens of thousands of bugs into popular programs such as file, binutils, and wireshark at the source code level.

\end{abstract}

\section{Motivation}
\label{sec:motivation}
\label{sec:motivation}

Bug-finding tools have been an active area of research for almost as long as computer programs have existed. 
Techniques such as abstract interpretation, fuzzing, and symbolic execution with constraint solving have been proposed, developed. and applied.
But evaluation has been a problem, as  ground truth is in extremely short supply.
Vulnerability corpora exist [cite SAMATE] but they are of limited utility and quantity.
These corpora fall into two categories: historic and synthetic.
Corpora built from historic vulnerabilities such as [cite ours and others] contain too few examples to be of much use.
These are closest to what we would want to have since the bug is embedded in real code and are often well annotated with precise information about where the bug manifests itself.
The author's own experience creating such a corpus was that it is a difficult and lengthy process; a corpus of only fourteen very well annotated historic bugs with triggering inputs took about six months to construct. 
In addition, public corpora have the disadvantage of already being released, and so we can expect tools to have been trained to detect those bugs.
Given the price tag of new exploitable bugs, which is widely understood to begin in the mid five figures [cite Hacking Team, http://www.wired.com/2015/07/hacking-team-leak-shows-secretive-zero-day-exploit-sales-work/], it is hard to find bugs for our corpus that have not been explicitly used to train existing bug-finding tools.
And, while synthetic code stocked with bugs, auto-generated by scripts, can provide large numbers of diverse and diagnostic examples, each is only a tiny program and the constructions are often considered oddball and unrepresentative of real code~\cite{kendra}.

In practice, a vulnerability discovery tool is typically evaluated by running it and seeing what it finds. 
Thus, one technique is judged superior if it finds more bugs than another.
While this state of affairs is perfectly understandable, given the scarcity of ground truth, it is an obstacle to science and progress in vulnerability discovery.
There is currently no way to measure fundamental figures of merit such as miss and false alarm rate for a bug finding tool.

We propose the following requirements for bugs in a vulnerability corpus, if it is to be useful for research, development, and evaluation.
Bugs must be
\begin{enumerate}
\item Cheap and plentiful
\item Span the execution lifetime of a program
\item Embedded in representative control and data flow
\item Come with an input that serves as an existence proof 
\item Manifest for a very small fraction of possible inputs
\end {enumerate}
The first requirement, if we can meet it, is highly desirable since it enables frequent evaluation and hill climbing. 
Corpora are more valuable if they are essentially disposable. 
The second and third of these requirements stipulate that bugs must be realistic.
The fourth means the bug is real and serious, and is a precondition for determining exploitability. 
The fifth is crucial.
Consider the converse: if a bug manifests for all or a large fraction of inputs it is trivially discoverable by simply running the program.

The approach we propose is to create a synthetic vulnerability via a few judicious and automated edits to the source code of a real program.
We will detail and give results for an implementation of this approach that satisfies all of the above requirements.
We call this implementation LAVA for Lincoln Application Vulnerability Assessment.    
A serious bug such as a buffer overflow can be injected by LAVA into a program like \verb+file+, which is 13K LOC, in about a 15 seconds.
LAVA bugs manifest all along the execution trace, in all parts of the program, shallow and deep, and make use of mostly completely normal data flow.
By construction, a LAVA bug comes with an input that triggers it, and no other input can have this effect upon the program.


\section{Scope}
\label{sec:scope}

We restrict our attention, with LAVA, to the injection of bugs into source code.
This makes sense given our interest in using it to assemble large corpora for the purpose of evaluating and developing vulnerability discovery techniques and systems.
Most automated vulnerability discovery systems work with source code [References?], and we can easily test binary analysis tools by simply compiling the source code with injected bugs.
Injecting bugs into binaries or byte code may also be possible using a similar approach, but we do not consider it here.

We want the injected bugs to be serious ones, i.e., potentially exploitable.
As a convenient proxy, our current focus is on injecting code that can result in out-of-bounds reads and writes.  
We produce a proof-of-concept input to trigger any bug we successfully inject, although we do not attempt to produce an actual exploit.

\section{LAVA Overview}
\label{sec:overview}

At a high level, VITAL adds bugs to programs in the following manner.

\begin {enumerate}
\item Identify source code locations where input bytes that do not determine control flow and are still close to their original form are available. 
We call these quantities DUAs, for dead, uncomplicated and available. 
\item Find potential attack points that are temporally after a DUA in the program trace.
Attack points are source code locations where a DUA might be used, if only it were available there as well, to make a program vulnerable. 
\item Add code to the program to make the DUA value available at the attack point and use it to create the vulnerability. 
\end{enumerate}

These three steps are depicted in Figure~\ref{lava-picture} and will be discussed in the following three sections,
which refer to the worked example in Figure~\ref{worked-example}.


\begin{figure} 
\centering
\includegraphics[width=3in]{worked-example.pdf}
\caption{
VITAL worked example
}
\label{fig:worked-example}
\end{figure}



\subsection {The DUA}

In a little more detail, the first step, in which DUAs are identified, is accomplished as follows.  

The program is executed under a dynamic taint analysis for a specific input.
That taint analysis has a few important features.
\begin{itemize}
\item Each individual byte in the input is given its own label.
Thus, if an internal program quantity is tainted and a direct copy of input bytes, then we can map that quantity back to a specific part of the input.  
\item The taint analysis operates, effectively, upon an LLVM version of the machine code for a program, including its libraries.
This means taint is propagated accurately and completely, even through esoteric x86 machine instructions like MMX and XMM.
\item The taint analysis keeps track of a \emph{set} of labels per input byte, meaning that it can represent computation that mixes input bytes.
\end{itemize}
We use the PANDA system to perform this taint analysis, with two crucial conceptual extensions in the form of taint-based measures.

The \emph{taint compute number} is a measure of how complicated a function of input bytes a tainted internal program quantity is.
TCN is illustrated in Figure~\ref{fig:taint-compute-number}; it simply tracks the depth of the tree of computation required to obtain 
a quantity from input bytes.
The smaller TCN is for a program quantity, the closer it is, computationally, to the input.
If TCN is 0, the quantity is a direct copy of input bytes.
The intuition behind this measure is that we need DUAs that are computationally close to the input in order to be able to use them with predictable results.
Note that TCN is not an ideal measure.
There are obviously situations in which the tree of computation is deep but the resulting DUA value is both completely predictable and has as much entropy as the original value.
However, TCN has the advantage that it is easy to compute.
Whenever the taint system needs to union label sets to represent computation, the TCN associated with the resulting set is one more than the max of those of the 
input sets.
TCN is an integer attached to the taint label set associated with each byte in a program.
In the worked example, $TCN(c)=1$ after line 1, since it is computed from quantities \verb+a+ and \verb+b+ which are directly derived from input.
Later, just before line 7 and after the loop, $TCN(c)=n+1$ because each iteration of the loop increases the depth of the tree of computation by one.  

\begin{figure}
\centering
\includegraphics[width=3in]{tcn.pdf}
\caption{Taint Compute Number (TCN)}
\label{fig:taint-compute-number}
\end{figure}

The other taint-based measure VITAL introduces is \emph{liveness}, which is associated with taint labels, i.e. the input bytes themselves.
This is a straightforward accounting of how many branches a byte in the input has been used to decide.
Thus, if a particular input byte label was never found in a taint label set associated with any byte used in a branch, it will have liveness of 0.
A DUA entirely consisting of bytes with 0 or very low liveness can be considered \emph{dead} in the sense that it has very little influence upon control flow for this program trace.
If one were to fuzz dead bytes, the program should be indifferent and execute the same trace.  
In the worked example, $LIV(0..3)$=1 after line 3, since \verb+a+ is a direct copy of input bytes 0..3.
After each iteration of the loop, the liveness of bytes 8..11 increase by one and so, after the loop, $LIV(8..11)=n$.

The combination of uncomplicated (low TCN) and dead (low liveness) is a powerful one for vulnerability injection.
The DUAs it identifies are internal program quantities that are often a direct copy of input bytes, and which can be set to any chosen value without sending the program along a different path.  
These make very good triggers for vulnerabilities.

In the worked example, bytes 0..3 and 8..11 are all somewhat live, because they have been seen to be used to decide branches.
Bytes 4..7, conversely, are uninvolved in any branches.
Variable \verb+a+ is therefore too live and \verb+c+, and \verb+n+ 

These have TCNs of 0 and 1, respectively, and liveness 0. 
These are tainted values that will make idea DUAs.

\subsection {The attack point}

Attack point identification is a function of the type of vulnerability to be injected.
If the goal is to inject a memory read overflow, then reads via pointer dereference, array index, and bulk memory copy, e.g., are reasonable attack points.  
If the goal is to inject divide-by-zero, then arithmetic operations involving division will be attacked. 
Alternately, the goal might be to control one or more arguments to a library function.
The only requirement is that the attack point be parameterized in order that it be attackable when the DUA is made available there. 

For instance, in Figure~\ref{worked-example}, on line 12, the call to \verb+memcpy+ can be attacked by introducing a data-flow relation between either \verb+x+ or \verb+y+ 
and any of the arguments to \verb+memcpy+.

\subsection {Data-flow bug injection}

The third and final step to VITAL bug injection is introducing a dataflow relationship between DUA and attack point.  
If the DUA is in scope at the attack point then it can simply be used at the attack point to introduce the vulnerability.
If it is not in scope, new code is added to copy the DUA off into a safe place (perhaps in a data structure or a global), and also retrieve it and  make use of it value at the attack point. 
However, in order to ensure that the bug only manifest itself very occasionally, we must introduce a guard requiring that the DUA match a specific value if it is to be used to manifest the vulnerability.

In the worked example in Figure~\ref{worked-example}, the DUA \verb+x+ is still in scope at the \verb+memcpy+ attack point and
 the only source code modification necessary is to make use of it to introduce the vulnerability if it matches a particular value.  











\section{Implementation}
\label{sec:implementation}
VITAL has four stages for injecting bugs.
\begin{enumerate}
\item Compile a version of the target program which has been instrumented with taint queries.
\item Run the instrumented version against various inputs and collect the results in a database.
\item Mine the database for potential injectable bugs.
\item Recompile the target with the relevant source code modifications, and test the bug to see if it was successfully injected.
\end{enumerate}

\subsection{Taint queries}
VITAL's taint queries rely on the PANDA dynamic analysis platform [ref], which is based on the QEMU whole-system emulator.
PANDA has a fast and robust dynamic taint analysis which uses LLVM lifting and Clang to ensure that taint propagates through uncommon instructions.
VITAL's Clang plugin, VITALtool, inserts taint queries at the arguments to each function call.
Each taint query consists of a ``hypervisor call'' which notifies PANDA to query the taint system for a specific source-level variable and record the results.
VITALtool also inserts hypervisor calls to record each attack point.

\subsection{Running the program}
Once the instrumented version has been compiled, we run it against a variety of inputs.
Since our approach to gathering data about the program is fundamentally dynamic, we must manually find a representative set of inputs to exercise as much of the program code as possible.
To run the program, we load it as a virtual CD into a PANDA virtual machine and send commands to QEMU over a virtual serial port.
As the hypervisor calls in the program execute, PANDA logs each one to a ``pandalog'' file.
The log connects source-level information like variable names and source file locations to the taint queries and attack points.

\subsection{Mine the database}
We then run a new analysis, FIB (Find Injectable Bugs), against the resulting logs.
FIB finds every possible injectable bug given our trace data and our analysis method.
It examines each taint query and finds instances where DUAs are available, as well as all attack points that occur after that DUA's availability temporally.
It produces a list of pairs of DUAs and attack points; each pair represents a triggering opportunity and a place to insert the bug.

\subsection{Inject the bugs}
For each DUA/attack point pair, we generate the C code which uses the DUA to trigger the bug.
At the point where the DUA is available, we shovel its contents aside into a global variable.
At the attack point, we insert code to trigger an out-of-bounds read or write if and only if the DUA has a magic value which corresponds to bytes in the input.
We then compile and test the modified program on a proof-of-concept input file (the original changed to include the magic value).


\section{Results}
\label{sec:results}
\label{section:results}

\begin{table*}[t]
\caption{LAVA Injection results for open source programs of various sizes}
\centering\footnotesize
\begin{tabular}{c|c|c|c|c|c|c|c|c|c|c}
        &         & Num       & Lines     &        &        & Potential & Validated  &         & Inj Time \\
Name    & Version & Src Files & C code    & N(DUA) & N(ATP) & Bugs      & Bugs       & Yield   & (sec)  \\\hline
file    & 5.22    & 19        & 10809     & 631    & 114    & 17518     & 774        & 38.7\%  & 16         \\  %    files verified.  sloc recomputed
%eog    & 3.4.2   &           & 22997     &        &        &           &            &         &         \\ 
readelf & 2.25    & 12        & 21052     & 3849   & 266    & 276367    & 1064       & 53.2 \% & 354     \\  % time verified.  files verified. sloc recomputed
bash    & 4.3     & 143       & 98871     & 3832   & 604    & 447645    & 192        & 9.6\%   & 153     \\  % time verified.  
tshark  & 1.8.2   & 1272      & 2186252   & 9853   & 1037   & 1240777   & 354        & 17.7\%  & 542     \\
\end{tabular} 
%For each, a single input file was used to perform a taint analysis with PANDA.
%Various program and dynamic trace statistics are reported as well as DUA, attack point (ATP), and yield (fraction of injected bugs that result in a segmentation violation).}
\label{table:insertion-results}
\end{table*}

We evaluated LAVA in three ways.
First, we injected large numbers of bugs into four open source programs: file, readelf (from binutils), bash, and tshark (the command-line version of the packet capture and analysis tool Wireshark).
For each of these, we report various statistics with respect to both the target program and also LAVA's success at injecting bugs.
Second, we evaluated the distribution and realism of LAVA's bugs by proposing and computing various measures.
Finally, we performed a preliminary investigation to see how effective existing bug-finding tools are at finding LAVA's bugs, by measuring the detection rates of an open-source fuzzer and a symbolic execution-based bug finder.

\subsection*{Counting Bugs}

Before we delve into the results, we must specify what it is we mean by an injected bug, and what makes two injected bugs distinct. Although there are many possible ways to define a bug, we choose a definition that best fits our target use case: two bugs should be considered different if an automated tool would have to reason about them differently. For our purposes, we define a bug as a unique pair $(DUA, attack point)$. Expanding this out, that means that the source file, line number, and variable name of the DUA, and the source file and line number of the attack point must be unique.

Some might object that this artificially inflates the count of bugs injected into the program, for example because it would consider two bugs distinct if they differ in where the file input becomes available to the program, even though the same file input bytes are used in both cases. 
But in fact these should be counted as different bugs: the data and control flow leading up to the point where the DUA occurs will be very different, and vulnerability discovery tools will have to reason differently about the two cases.

\subsection{Injection Experiments}
\label{sec:results:subsec:injection}

The results of injecting bugs into open source programs are summarized in Table~\ref{table:insertion-results}.
In this table, programs are ordered by size, in lines of C code, as measured by David Wheeler's \verb+sloccount+.
A single input was used with each program to measure taint and find injectable bugs.
The input to \verb+file+ and \verb+readelf+ was the program \verb+ls+.
The input to \verb+tshark+ was a 16K packet capture file from a site hosting a number of such examples.  % [ http://www.stearns.org/toolscd/current/pcapfile/README.ethereal-pcap.html ]
%The input to \verb+eog+ was ... 
The input to \verb+bash+ was a 124-line shell script written by the authors.
% TRL -- removed this since table was too big and I wanted to add bug inj time
%The number of sequential and unique basic blocks in the PANDA trace for each program are reported as N(BBs) and N(BBu).
$N(DUA)$ and $N(ATP)$ are the number of DUAs and attack points collected by the \verb+FIB+ analysis.
Note that, in order for a DUA or attack point to be counted, it must have been deemed viable for some bug, as described in Section~\ref{sec:mining}.
The columns \emph{Potential Bugs} and \emph{Validated Bugs} in Table~\ref{table:insertion-results} give the numbers of both potential bugs found by \verb+FIB+, but also those verified to actually return exitcodes indicating a buffer overflow (-11 for segfault or -6 for heap corruption) when run against the modified input.
The penultimate column in the table is \emph{Yield} which is the fraction of potential bugs what were tested and determined to be actual buffer overflows.
The last column gives the time required to test a single potential bug injection for the target.


Exhaustive testing was not possible for a number of reasons.
Larger targets had larger numbers of potential bugs and take longer to test; for example, \verb+tshark+ has over a million bugs and took almost 10 minutes to test.
This is because testing requires not only injecting a small amount of code to add the bug, but also recompiling and running the resulting program.
For many targets, we found the build to be subtly broken so that a \verb+make clean+ was necessary to pick up the bug injection reliably, which further increased testing time.
Instead, we attempted to validate 2000 potential bugs chosen uniformly at random for each target.
Thus, when we report, in Table~\ref{table:insertion-results} that for \verb+tshark+, the yield is 17.7\%, this is because 306 out of 2000 bugs were found to be valid.

As the injected bug is designed to be triggered only if a particular set of four bytes in the input is set to a magic value, we tested with both the original input and with the modified one that contained the trigger. 
We did not encounter any situation in which the original input triggered a crash.

\begin{table}[b]
\caption{Yield as a function of both $mLIV$ and $mTCN$}
\centering\footnotesize
\begin{tabular}{l|l|l|l|l} 
       & \multicolumn{3}{c}{$mLIV$} &  \\  
$mTCN$ &         $[0,10)$ & $[10,100)$ & $[100,1000)$ & $[1000,+\inf]$ \\  \hline 
$[0,10)$ &       51.9\%   & 22.9\%     & 17.4\%       & 11.9\%          \\
$[10,100)$ &     --       & 0          & 0            & 0     \\
$[100,+\inf]$ &  --       & --         & --           & 0     \\ 
\end{tabular}
%Yield is highest for DUAs with low values for both of these measures, i.e., that are both a relatively uncomplicated function of input bytes and also that derive from input bytes involved in deciding fewer branches.
%Cells for which there were no samples are indicated with the contents '--'.}
\label{table:yield-breakdown}
\end{table}

Yield varies considerably from less than 10\% to over 50\%.
To understand this better, we investigated the relationship between our two taint-based measures and yield.
For each DUA used to inject a bug, we determined $mTCN$, the maximum TCN for any of its bytes and $mLIV$, the maximum liveness for any label in any taint label set associated with one of its bytes.  
More informally, $mTCN$ represents how complicated a function of the input bytes a DUA is, and $mLIV$ is a measure of how much the control flow of a program is influenced by the input bytes that determine a DUA.
Table~\ref{table:yield-breakdown} is a two-dimensional histogram with the bins for $mTCN$ intervals along the vertical axis and bins for $mLIV$ along the horizontal axis.
The top-left cell of this table represents all bug injections for which $mTCN<10$ and $mLIV<10$, and the bottom-right cell is all those for which $mTCN>=1000$ and $mLIV>=1000$.
Recall that when  $mTCN=mLIV=0$, the DUA is not only a direct copy of input bytes, but those input bytes have also not been observed to be used in deciding any program branches. 
As either $mTCN$ or $mLIV$ increase, yield deteriorates.  
However, we were surprised to observe that $mLIV$ values of over 1000 still gave yield in the 10\% range.

\subsection{Bug Distribution}

It would appear that LAVA can inject a very large number of bugs into a program.
If we extrapolate from yield numbers in Table~\ref{table:insertion-results}, we estimate there would be almost 400,000 real bugs if all were tested.
But how well distributed is this set of bugs? 
For programs like \verb+file+ and \verb+bash+, between 11 and 44 source files  are involved in a potential bug.
In this case, the bugs appear to be fairly well distributed, as those numbers represent 58\% and 31\% of the total for each, respectively.
On the other hand, \verb+readelf+ and \verb+tshark+ fare worse, with only 2 and 122 source files found to involve a potential bug for each (16.7\% and 9.6\% of source files).
For \verb+tshark+, much of the code for which is devoted to parsing esoteric network protocols, coverage is probably the issue since we use only a single input.
Similiary, we only use a single hand-written script with \verb+bash+, with little attempt to cover a majority of language features.
We are unsure why so few of the source files in \verb+readelf+ involve a potential bug.
 

\subsection{Bug Realism}

%\todo[inline]{Ricky: I think it would be particularly convincing to have a "case study" of an actual CVE that exhibits data flow properties similar to the kind of bug LAVA injects.  Or maybe a bug that adds unchecked tainted data to a pointer.} 

\begin{figure}
\centering
\includegraphics[width=3in]{trace-dua-atp.png}
\caption{A cartoon representing an entire program trace, annotated with instruction count at which DUA is siphoned off to be used, $I(DUA)$, attack point where it is used, $I(ATP)$, and total number of instructions in trace, $I(TOT)$.}
\label{fig:dua-atp-trace}
\end{figure}

The intended use of the bugs created by this system is as ground truth for development and evaluation of vulnerability discovery tools and techniques. 
Thus, it is crucial that they be realistic in some sense.  
Realism is, however, difficult to assess.

Because this work is, to our knowledge, the first to consider the problem of fully automated bug injection, we are not able to make use of any standard measures for bug realism.
Instead, we devised our own measures, focusing on features such as how well distributed the malformed data input and trigger points were in the program's execution, as well as how much of the original behavior of the program was preserved.

We examined three aspects of our injected bugs as measures of realism. 
The first two are DUA and attack point position within the program trace, which are depicted in Figure~\ref{fig:dua-atp-trace}.
That is, we determined the fraction of trace instructions executed at the point the DUA is siphoned off and at the point it is used to attack the program by corrupting an internal program value.

Histograms for these two quantities, $I(DUA)$ and $I(ATP)$, are provided in Figures~\ref{fig:dua-hist} and~\ref{fig:atp-hist}, where counts are for all potential bugs in the LAVA database for all five open source programs. 
DUAs and attack points are clearly available at all points during the trace, although there appear to be more at the beginning and end.
This is important, since bugs created using these DUAs have entirely realistic control and data-flow all the way up to $I(DUA)$.
Therefore, vulnerability discovery tools will have to reason correctly about all of the program up to $I(DUA)$ in order to correctly diagnose the bug.

Our third metric concerns the portion of the trace \emph{between} the $I(DUA)$ and $I(ATP)$.
This segment is of particular interest since LAVA currently makes data flow between DUA and attack point via a pair of function calls.
Thus, it might be argued that this is an unrealistic portion of the trace in terms of data flow.
The quantity $I(DUA)/I(ATP)$ will be close to 1 for injected bugs that minimize this source of unrealism.
This would correspond to the worked example in Figure~\ref{fig:worked-example}; the DUA is still in scope when, a few lines later in the same function, it can be used to corrupt a pointer.
No abnormal data flow is required.
The histogram in Figure~\ref{fig:rdf-hist} quantifies this effect for all potential LAVA bugs, and it is clear that a large fraction have $I(DUA)/I(ATP) \approx 1$, and are therefore highly realistic by this metric.

\begin{figure}
\centering
\includegraphics[width=3in]{dua.pdf}
\caption{Normalized DUA trace location}
\label{fig:dua-hist}
\end{figure}

\begin{figure}
\centering
\includegraphics[width=3in]{atp.pdf}
\caption{Normalized ATP trace location}
\label{fig:atp-hist}
\end{figure}

\begin{figure}
\centering
\includegraphics[width=3in]{rdf.pdf}
\caption{Fraction of trace with perfectly normal or realistic data flow, $I(DUA)/I(ATP)$}
\label{fig:rdf-hist}
\end{figure}




\subsection{Vulnerability Discovery Tool Evaluation}

We ran two vulnerability discovery tools on LAVA-injected bugs to investigate their use in evaluation.

\begin{enumerate}
\item Coverage guided fuzzer (referred to as FUZZER)
\item Symbolic execution + SAT solving (referred to as SES)
\end{enumerate}

These two, specifically, were chosen because fuzzing and symbolic execution are extremely popular techniques for finding real-world bugs.
FUZZER and SES are both state-of-the-art, high-profile tools. 
For each tool, we expended significant effort to ensure that we were using them correctly.
This means carefully reading all documentation, blog posts, and email lists.
Additionally, we constructed tiny example buggy programs and used them to verify that we were able to use each tool at least to find known easy bugs.  
For some tools we had to resort to reading source code in order to determine the proper command-line incantations to employ. 
We did not contact tool authors for guidance in tool use, as this seemed likely to bias results.

Note that the names of tools under evaluation are being withheld in reporting results.
Careful evaluation is a large and important job, and we would not want to give it short shrift, either in terms of careful setup and use of tools, or in presenting and discussing results.
Our intent, here, is to determine if LAVA bugs \emph{can be used} to evaluate bug finding systems. 
It is our expectation that, in future work either by ourselves or others, full and careful evaluation of real, named tools will be performed using LAVA.
While that work is outside the scope of this paper, we hope to indicate that it should be both possible and valuable. 
Additionally, it is our plan and hope that LAVA bugs will be made available in quantity and at regular refresh intervals, for self-evaluation and hill climbing.

The first corpus we created, \emph{LAVA-1}, used the \verb+file+ target, the smallest of those programs into which we have injected bugs.
This corpus consists of sixty-nine buffer overflow bugs injected into the source with LAVA, each on a different branch in a \verb+git+ repository with a fuzzed version of the input verified to trigger a crash checked in along with the code.
Two types of buffer overflows were injected, each of which makes use of a single 4-byte DUA to trigger and control the overflow.

\begin{enumerate}
    \item \textbf{Knob-and-trigger}. 
In this type of bug, two bytes of the DUA (the \emph{trigger}) are used to test against a magic value to determine if the overflow will happen.
The other two bytes of the DUA (the \emph{knob}) determine how much to overflow. 
Thus, these bugs manifest if a 2-byte unsigned integer in the input is a particular value but only if another 2-bytes in the input are big enough to cause trouble. 
    \item \textbf{Range}. 
These bugs trigger if the magic value is simply in some range, but also use the magic value to determine how much to overflow.
The magic value is a 4-byte unsigned integer and the range varies.  
\end{enumerate}

These bug types were designed to mirror real bugs patterns.  
In knob-and-trigger bugs, two different parts of the input are used in different ways to determine the manifestation of the bug.  
In range bugs, rather than triggering on a single value out of $2^{32}$, the size of the haystack varies.
Note that a range of $2^0$ is equivalent to the bug presented in Figure~\ref{src:dua-use}.

\begin{table}[h]
\caption{Percentage of bugs found in \emph{LAVA-1} corpus} %. % as a function of $Bug Type$ and $Tool Type$.  
\centering\footnotesize
\begin{tabular}{l|l|l|l|l|l|l} 
Tool   &                     \multicolumn{6}{|c}{Bug Type}                           \\  \hline  
         &                     \multicolumn{5}{|c|}{Range}                   &     \\   
         &    $2^0$   & $2^7$       & $2^{14}$     & $2^{21}$   & $2^{28}$     & KT   \\  \hline 
FUZZER &    0       & 0           & 0            & 7\%        & 58\%         & 0         \\
%$CSA_1$ &    0       & 0           & 0            & 0          & 0            & 0         \\
SES    &    8\%     & 0           & 9\%          & 21\%       & 0            & 10\%         \\
\end{tabular}
%\caption{Percentage of bugs found in \emph{LAVA-1} corpus. % as a function of $Bug Type$ and $Tool Type$.  
%FUZZER proved to be an effective vulnerability finding tool.
%However, it was only able to find bugs that allowed an enormous range of possible inputs to trigger the bug.}
\label{table:eval1-file}
\end{table}

The results of this evaluation are summarized in Table~\ref{table:eval1-file}.
Five different ranges were employed: $2^0, 2^7, 2^{16}, 2^{21}, 2^{28}$. 
We examined all output from both tools.
FUZZER ran for one hour on each bug and found 7\% of those with a range of $2^{21}$ and 58\% of those with a range size of $2^{28}$.
SES ran for five hours on each bug, and found several bugs in all categories except the $2^7$ and $2^{28}$ ranges.

The results for the \emph{LAVA-1} corpus seem to accord well with how these tools work.
FUZZER uses the program largely as a black box, randomizing individual bytes, and guiding exploration with coverage measurements.
Bugs that trigger if and only if a four-byte extent in the input is set to a magic value are unlikely to be discovered in this way.
Given time, FUZZER finds bugs that trigger for ranges of a million or more bytes. 
Note that for many of these LAVA bugs, when the range is so large, discovery is possible by simply fuzzing every byte in the input a few times.  
These bugs may, in fact, be trivially discoverable with a regression suite for a program like \verb+file+ that accepts arbitrary file input\footnote{In principle, anyway. In practice \texttt{file}'s test suite consists of just 3 tests, none of which trigger our injected bugs.}.
By contrast, SES is able to find both knob-and-trigger bugs and different ranges, and the size of the range does not affect the number of bugs found.
This is because it is no more difficult for a SAT solver to find a satisfying input for a large range or a small range; rather, the number of bugs found is limited by how deep into the program the symbolic execution can go.

Note that having each bug in a separate copy of the program means that for each run of a bug finding tool, only one bug is available for discovery at a time.  
This is one kind of evaluation, but it seems to disadvantage tools like FUZZER and SES, which appear to be designed to work for a long time (10s of hours) on a single program that may contain multiple bugs. 

Thus, we created a second corpus, \emph{LAVA-M}, in which we injected more than one bug at a time into the source code.
We chose four programs from the \verb+coreutils+ suite that took file input: \verb+base64+, \verb+md5sum+, \verb+uniq+, and \verb+who+.
Into each, we injected as many verified bugs as possible.
Because the \verb+coreutils+ programs are quite small, and because we only used a single input file for each to perform the taint analysis, the total number of bugs injected into each program was generally quite small.
The one exception to this pattern was the \verb+who+ program, which parses a binary file with many dead or even unused fields, and therefore had many DUAs available for bug injection.
We were not able to inject multiple bugs of the two types described above (knob-and-trigger and range), as interactions became a problem, and so all bugs were of the type in Figure~\ref{src:dua-use}, which trigger for only a single setting of four input bytes.  
The \emph{LAVA-M} corpus, therefore, is four copies of the source code for \verb+coreutils+ version 8.24.
One copy has 44 bugs injected into \verb+base64+, and comes with 44 inputs known to trigger those bugs individually.
Another copy has 57 bugs in \verb+md5sum+, another has 28 bugs in \verb+uniq+.
Finally, there is a copy of \verb+coreutils+ with 2136 bugs existing all at once and individually expressible, in \verb+who+.


\begin{table}[h]
\caption{Bugs found in \emph{LAVA-M} corpus by tool type}
\centering\footnotesize
\begin{tabular}{l|c|c|c|c} 
\multirow{2}{*}{Tool Name} & \multirow{2}{*}{Total Bugs} & \multicolumn{3}{c}{Unique Bugs Found} \\
              &            & FUZZER       & SES        & Combined \\ \hline 
\verb+uniq+   &    28      & 7            & 0          & 7               \\
\verb+base64+ &    44      & 7            & 9          & 14               \\
\verb+md5sum+ &    57      & 2            & 0          & 2               \\
\verb+who+    &    2136    & 0            & 18         & 18               \\
Total         &    2265    & 16           & 27         & 41               \\
\end{tabular}
\label{table:tool-eval-results-coreutils}
\end{table}

We ran FUZZER and SES against each program in \emph{LAVA-M}, with 5 hours of runtime for each program on machines with essentially unlimited RAM.
\verb+md5sum+ ran with the \verb+-c+ argument, as if it were checking digests in a symbolic file.
\verb+base64+ ran with the \verb+-d+ argument, to decode base 64.

SES found no bugs in \verb+uniq+ or \verb+md5sum+.
In \verb+uniq+, we believe this is because the control flow is too unconstrained.
In \verb+md5sum+, SES failed to execute any code past the first instance of the hash function.
\verb+base64+ and \verb+who+ both turn out more successful for SES.
The tool finds 9 bugs in \verb+base64+ out of 44 inserted; these include both deep and shallow bugs, as base64 is such a simple program to analyze.

SES's results are a little more complicated for \verb+who+.
All of the bugs it finds for \verb+who+ use one of two DUAs, and all of them occur very early in the trace.
One artifact of our method for injecting multiple bugs simultaneously is that multiple bugs share the same attack point.
It is debatable how realistically this aspect represents real bugs.
In practice, it means that SES can only find one bug per attack point, as finding an additional bug at the same ATP does not necessarily require covering new code.
LAVA could certainly be changed to have each bug involve new code coverage.
SES could also be improved to find all the bugs at each attack point, which means generating multiple satisfying inputs for the same set of conditions.

FUZZER found bugs in all utilities except \verb+who+\footnote{In fact, we allowed FUZZER to continue running after 5 hours had passed; it managed to find a bug in \texttt{who} in the sixth hour.}.
Unlike SES, the bugs were fairly uniformly distributed throughout the program, as they depend only on guessing the correct 4-byte trigger at the right position in the input file.

FUZZER's failure to find bugs in \verb+who+ is surprising.
We speculate that the size of the seed file (the first 768 bytes of a \verb+utmp+ file) used for the fuzzer may have been too large to effectively explore through random mutation, but more investigation is necessary to pin down the true cause.
Indeed, tool anomalies of this sort are exactly the sort of thing one would hope to find with LAVA, as they represent areas where tools might make easy gains.

We note that the bugs found by FUZZER and SES have very little overlap (only 2 bugs where found by both utility).
This is a very promising result for LAVA, as it indicates that the kinds of bugs created by LAVA are not tailored to a particular bug finding strategy.

%We also evalued a second commercial static analyzer, referred to in this text as $CSA_2$. Because we were only allowed a limited number of analysis runs, we opted to create a single version of \verb+who+ from the GNU coreutils that contained 60 injected, verified bugs. This was fed to $CSA_2$, generating 117 alerts; we then went through its alerts by hand to see which ones referenced our injected bugs.

%Of the 117 alerts, 17 referred to code we had injected into \verb+who+. We did not investigate the other 100 further; presumably these are some mix of true and false positives in the original source code to \verb+who+. The 17 that were specific to our injected bugs were examined in more detail. Fourteen were found to be essentially false positives: they referred to innocuous artifacts of the injection process. For example, to guard against introducing pointer errors, we emit code like \verb+if(p) { ... }+; however, the static analysis software took this as proof that \verb+p+ might be NULL, and then warned us that it was dereferenced elsewhere before the NULL check. After discounting such artifacts, we found that $CSA_2$ correctly identified three of our injected bugs, labeling them all as ``Out-of-bounds access''.


\section{Related Work}
\label{sec:relwork}
The design of LAVA is driven by the need for bug corpora that are a) dynamic (can produce new bugs on demand), b) realistic (the bugs occur in real programs and are triggered by the program's normal input), and c) large (consist of hundreds of thousands of bugs). 
In this section we survey existing bug corpora and compare them to the bugs produced by LAVA.

The need for realistic corpora is well-recognized. 
Researchers have proposed creating bug corpora from student code~\cite{Spacco:2005}, drawing from existing bug report databases~\cite{Lu:2005,Meftah:2005},
and creating a public bug registry~\cite{Foster:2005}. 
Despite these proposals, public bug corpora have remained static and relatively small.

The earliest work on tool evaluation via bug corpora appears to be by Wilander and Kamkar, who created a synthetic testbed of 44 C function calls~\cite{Wilander:2002} and 20 different buffer overflow attacks~\cite{Wilander:2003} to test the efficacy of static and dynamic bug detection tools, respectively. 
These are synthetic test cases, however, and may not reflect real-world bugs. 
In 2004, Zitser et al.~\cite{Zitser:2004} evaluated static buffer overflow detectors; their ground truth corpus was painstakingly assembled by hand over the course of six months and consisted of 14 annotated buffer overflows with triggering and non-triggering inputs as well as buggy and patched versions of programs; these same 14 overflows were later used to evaluate dynamic overflow detectors~\cite{Zhivich:2005}.
Although these are real bugs from actual software, the corpus is small both in terms of the number of bugs (14) but also in terms of program size.
Even modest sized programs like \verb+sendmail+ were too big for some of the static analyzers and so much smaller models capturing the essense of  each bug were constructed in a few hundred lines of excerpted code.  

The most extensive effort to assemble a public bug corpus comes from the NIST Software Assurance Metrics And Tool Evaluation (SAMATE)
project~\cite{Kass:2005}. 
Their evaluation corpus inclues Juliet~\cite{Juliet:2012}, a collection of 86,864 synthetic C and Java programs that exhibit 118 different CWEs; each program, however, is relatively short and has uncomplicated control \& data flow. 
The corpus also includes the IARPA STONESOUP data set~\cite{SAMATE:2014}, which was developed in support of the STONESOUP vulnerability mitigation project.
The test cases in this corpus consist of 164 small snippets of C and Java code, which are then spliced into program to inject a bug. 
The bugs injected in this way, however, do not use the original input to the program (they come instead from extra files and environment variables added to the program), and the data flow between the input and the bug is quite short.

Finally, the general approach of automatic program transformation to introduce errors was also used by Rinard et al.~\cite{Rinard:2005}; the authors systematically modified the termination conditions of loops to introduce off-by-one errors in the Pine email client to test whether software is still usable in the presence of errors once sanity checks and assertions are removed.



\section{Limitations and Future Work}
\label{sec:future}
%Limitations:
%- C code only
%- Dataflow is... manufactured, maybe not reflective
%- Limited class of bugs (pointer corruptions)

Future work for VITAL largely involves making the generated corpora look more like the bugs that are found in real programs. The most obvious limitation is that it only works on C source code. In principle, our approach would work for any source language with a usable source-to-source rewriting framework. In addition, VITAL currently injects only pointer corruption bugs. But our taint-based analysis overcomes the crucial first hurdle to injecting any kind of bug: making sure that attacker-controlled data can be used in the bug's potential exploitation. As a result, the addition of other classes of bugs, such as temporal safety bugs (use-after-free) and meta-character bugs (e.g. format string) should also be injectable using our approach. Finally, there is work to be done in making VITAL's data flow more realistic, although even in its current state, the vast majority of the execution of the modified program is realistic. This execution includes the dataflow that leads up to the capture of the DUA, which can often be nontrivial.

\section{Conclusion}
In this paper, we have introduced LAVA, a fully automated system that can inject large numbers of realistic bugs into C programs.
LAVA has already been used to introduce over 2000 realistic buffer overflows into open-source Linux C programs of between 10,000 and 2 million lines of code.  
The taint-based measures employed by LAVA to identify attacker-controlled data for use in creating new vulnerabilities should be usable to inject other classes of vulnerabilities than the buffer overflow we demonstrate here, and we will pursue that actively.  
We believe LAVA will be of immense value as an on-demand source ground truth corpora of very large size.
The availability of these corpora should energize research and development into automated vulnerability discovery tools and techniques, as well as the evaluation thereof.


%LAVA is fast, injecting a new buffer overflow into a program like \verb+file+ in less than 20 seconds.



\bibliographystyle{plain}
\bibliography{biblio}

\end{document}
